\documentclass{jarticle}
\usepackage{amsmath, amssymb, amsthm, amscd}
\usepackage[all]{xy}


\usepackage{inconsolata}
\usepackage{verbatim}
\usepackage[dvipdfmx]{hyperref}
\usepackage{framed}

\newcommand{\code}[1]{\texttt{#1}}

\newcommand{\mathint}[0]{\mathbb{Z}}

\newcommand{\category}[1]{\mathcal{#1}}

\title{圏これ}
\author{Kinebuchi Tomohiko}
\date{\today}

\begin{document}

\section{極限, 余極限}

\url{https://ncatlab.org/nlab/show/limit#in_}

ここの書き方がよく分からない

極限って図式に対して決まるんじゃなかったっけ?
→ 分かるかボケ

\url{https://ncatlab.org/nlab/show/limits+and+colimits+by+example#limcoliminset}




\end{document}
